\subsection{Energia total}
Embora para as outras integrais primeiras a coisa não mude muito, para a energia total precisamos ser cuidadosos. Vamos seguir a mesma ideia de antes de aplicar transformações $\vet q \mapsto \alpha^{-1}$ e $\vet p \mapsto \beta \vet p$. Temos o $\beta$:
\begin{equation}
    \beta^2 = \dfrac{\tilde E - V_\varepsilon(\alpha^{-1} \vet q_0) - S_2}{S_1}.
\end{equation}
A restrição para a existência de $\beta$ fica parecida:
\begin{equation}
    \tilde E - V_{\varepsilon}(\alpha^{-1} \vet q_0) + \dfrac{||\tilde{\vet P}||^2}{2M} + \dfrac{1}{2} \alpha^2 \tilde \sigma < 0.
\end{equation}
Agora, porém, não temos como resolver o sistema explicitamente.

Para facilitar, vamos tomar $\alpha = \frac{1}{h \varepsilon}$. Temos então:
\begin{equation}
    V_\varepsilon(\alpha^{-1} \vet q_0) 
    = V_\varepsilon(h \varepsilon \vet q_0) 
    = \varepsilon^{-1} V_1(h \vet q_0),
\end{equation}
e logo
\begin{equation}
    - \tilde E + \varepsilon^{-1} V_1(h \vet q_0) + \dfrac{||\tilde{\vet P}||^2}{2M} + \dfrac{\tilde \sigma}{2 \varepsilon^2 h^2} < 0.
\end{equation}

Seja $f(h)$ a expressão do lado esquerdo da inequação anterior. Embora não tenhamos uma parábola como no caso homogêneo, ainda é possível analisar a existência de soluções. No caso em que o momento angular total é nulo, temos $\tilde E - \frac{1}{2M} ||\tilde{\vet P}||^2 > \varepsilon^{-1} V_1 (h \vet q_0)$, o que é trivial no caso em que $\tilde E \geq 0$ e o momento linear total é nulo, e existe quando $\tilde E < 0$ e/ou temos momento linear uma vez que os corpos não comecem todos parados.

No caso em que temos momento angular total não nulo, é preciso lidar com o $h$ no denominador. Podemos reescrever a inequação como:
\begin{equation}
    \varepsilon^{-1} V_1 (h \vet q_0) + \dfrac{\tilde \sigma}{2 \varepsilon^2 h^2} < \tilde E - \dfrac{||\tilde{\vet P}||^2}{2M}.
\end{equation}
Para começar, suponha que o momento linear total é nulo. Vamos multiplicar ambos os lados por $h^2$ e rearranjar os termos:
\begin{equation}
    2 h^2 \varepsilon^2 (\tilde E - \varepsilon^{-1} V_1 (h \vet q_0))
    > \tilde \sigma
\end{equation}
Se tomamos $h = \frac{1}{\varepsilon \sqrt(2 I_0)}$, a expressão ser válida significa que vale tambémn a desigualdade de Sundman, então podemos começar por aqui. 

[AINDA NAO CONSEGUI MOSTRAR EXISTÊNCIA, TÁ OSSO]

Garantida a existência, podemos aplicar algum método iterativo. Por exemplo, podemos derivar $f$:
\begin{equation}
    f'(h) = \varepsilon^{-1} \der{V_1 (h, \vet q_0)}{h} - \dfrac{\tilde \sigma}{\varepsilon^2 h^3}.
\end{equation}
Veja que:
\begin{equation}
    \der{V_1(h, \vet q_0)}{h} = h \sum_{a < b} G m_a m_b \dfrac{r_{ab}^2}{(h^2 r_{ab}^2 + 1)^{3/2}} =: h \Phi_h(\vet q_0),
\end{equation}
e logo
\begin{equation}
    f'(h) = h \varepsilon^{-1} \Phi_h(\vet q_0) - \dfrac{\tilde \sigma}{\varepsilon^2 h^3}.
\end{equation}

Com isso, tomando um chute inicial $h_0$ (que pode ser, por exemplo, o valor que usaríamos se o potencial não fosse amortecido), podemos aplicar o Método de Newton:
\begin{equation}
    h_{i+1} = h_i - \dfrac{f(h_i)}{f'(h_i)}.
\end{equation}