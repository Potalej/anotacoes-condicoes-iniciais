\subsection{Método direto}
Há, no entanto, uma forma de unir todas as transformações adequadamente. De fato, basta tomar
\begin{align}
    \tilde{\vet q_a} &= \alpha^{-1} \vet q_a, \\
    \tilde{\vet p_a} &= \beta\left(\vet p_a - \dfrac{m_a}{M} \left(\vet P - \beta^{-1} \tilde{\vet P}\right) - m_a \vet q_a \times \vet \omega\right), \\
    \bm I_T \vet \omega &= \vet J - \alpha \beta^{-1} \tilde{\vet J}.
\end{align}

Como $\alpha$ impacta fortemente no valor de $\beta$, vamos obter $\beta$ primeiro. Veja que temos o seguinte para a energia total:
\begin{equation}\label{eq:t_til_e_til_v0}
    T(\tilde{\vet p}) = \tilde E - \alpha V_0.
\end{equation}
Vamos expandir $T(\tilde{\vet p})$:
\begin{align*}
    T(\tilde{\vet p}) 
    &= \sum_{a=1}^{N} \dfrac{\beta^2}{2m_a} \norma{\vet p_a - \dfrac{m_a}{M}\left(\vet P - \beta^{-1} \tilde{\vet P}\right) - m_a \vet q_a \times \bm I_{T}^{-1} (\vet J - \alpha \beta^{-1} \tilde{\vet J})}^2.
\end{align*}
Podemos separar $\tilde T$ da seguinte forma:
\begin{equation}\label{eq:separacao_t_til}
    \tilde T = \beta^2 S_1 + S_2,
\end{equation}
onde:
\begin{align*}
    S_1 &= \sum_{a=1}^N \dfrac{1}{2m_a} \norma{\vet K_1^a}^2
    = \sum_{a=1}^N \dfrac{1}{2m_a} \norma{\vet p_a - \dfrac{m_a}{M} \vet P - m_a \vet q_a \times (\bm I_T^{-1} \vet J)}^2, \\
    S_2 &= \sum_{a=1}^N \dfrac{1}{2m_a} \norma{\vet K_2^a}^2 = \sum_{a=1}^N \dfrac{1}{2 m_a} \norma{\dfrac{m_a}{M} \tilde{\vet P} + \alpha m_a \vet q_a \times (\bm I_T^{-1} \tilde{\vet J})}^2,
\end{align*}
e não é difícil ver que $\sum_{a=1}^N m_a^{-1} \prodint{\vet K_1^a}{\vet K_2^a} = 0$.

De fato, podemos melhorar bastante as expressões de $S_1$ e $S_2$. Com algumas contas chegamos no seguinte para $S_1$:
\begin{equation}
    S_1 = T_0 - \dfrac{\norma{\vet P_0}^2}{2 M} + \dfrac{1}{2} \sum_{a=1}^N m_a \norma{\vet q_a \times \vet \omega_0}^2 + \prodint{\vet J_0}{\bm I_T^{-1} \vet J_0}.
\end{equation}
A expressão que ainda possui um somatório pode ser melhorada:
$$
\norma{\vet q_a \times \vet \omega}^2 = \norma{\vet q_a}^2 \norma{\vet \omega}^2 - \prodint{\vet q_a}{\vet \omega}^2
=
\vet \omega^T (\norma{\vet q_a}^2 Id_3 - \vet q_a \vet q_a^T) \vet \omega.
$$

Observe também que o tensor de inércia $\bm I_a$ do corpo $a$ pode ser escrito como:
$$
\bm I_a = m_a (\vet q_a \vet q_a^T - \norma{\vet q_a}^2 Id_3),
$$
então temos para a soma:
$$
\sum_{a=1}^N m_a \norma{\vet q_a \times \vet \omega_0}^2
= - \vet \omega_0^T \sum_{a=1}^N \bm I_a \vet \omega_0
= - \vet \omega_0^T \bm I_T \vet \omega_0.
$$
Como $\vet \omega_0 = \bm I_T^{-1} \vet J_0$, temos
$$
= - \vet J_0^T (\bm I_T^{-1})^T \bm I_T \bm I_T^{-1} \vet J_0
= - \vet J_0^T (\bm I_T^{-1})^T \vet J_0.
$$
Assim, $S_1$ pode ser escrito como:
\begin{equation}\label{eq:S_1}
    S_1 = T_0 - \dfrac{\norma{\vet P_0}^2}{2 M} + \dfrac{1}{2} \prodint{\vet J_0}{\bm I_T^{-1} \vet J_0}.
\end{equation}

Através da mesma propriedade é fácil ver também que:
\begin{equation}\label{eq:S_2}
    S_2 = \dfrac{||\tilde{\vet P}||^2}{2M} - \dfrac{\alpha^2}{2} \prodint{\tilde{\vet J}}{\bm I_T^{-1} \tilde{\vet J}} = \dfrac{||\tilde{\vet P}||^2}{2M} + \dfrac{\alpha^2}{2} \tilde{\sigma}.
\end{equation}

Voltando às constantes, $\beta$ fica determinado uma vez que escolhemos $\alpha$. Porém, $\alpha$ não pode ser qualquer um. Juntando as equações (\ref{eq:t_til_e_til_v0}) e (\ref{eq:separacao_t_til}) temos:
\begin{equation}\label{eq:beta1}
    \beta^2 = \dfrac{\tilde E - \alpha V_0 - S_2}{S_1}.
\end{equation}
Como $S_1, S_2, - V_0 > 0$, precisamos escolher $\alpha$ de acordo com $\tilde E$ de modo a garantir que exista tal $\beta \in \R$. Para isso, basta que $\tilde E - \alpha V_0 - S_2 > 0$. Expandindo:
\begin{equation}\label{eq:inequacao_delta1}
    -\tilde E + \alpha V_0 + \dfrac{||\tilde{\vet P}||^2}{2M} + \dfrac{\alpha^2}{2} \tilde \sigma < 0,
\end{equation}
o que novamente é uma inequação de segundo grau para $\alpha$ com coeficiente quadrático positivo, então a solução, se houver, será o intervalo aberto entre as duas raízes da equação. Como na desigualdade de Sundman, aqui o discriminante nos impõe uma condição:
\begin{equation}
    \Delta_1 = V_0^2 - \tilde \sigma (M^{-1} ||\tilde{\vet P}||^2 - 2 \tilde E) \geq 0
\end{equation}
\begin{equation}\label{eq:restricao_delta1}
    \Rightarrow
    V_0^2 \geq \tilde{\sigma} (M^{-1} ||\tilde{\vet P}||^2 - 2 \tilde E).
\end{equation}

Aqui, $\tilde \sigma M^{-1} ||\tilde{\vet P}||^2 \geq 0$, mas a existência de $\alpha$ e $\beta$ depende de $V_0^2$ vencer a energia total desejada. Vale também ressaltar que o tensor de inércia total é também homogêneo de grau 2, então não adianta tentar reescalar as posições iniciais sorteadas por algum fator, como no outro caso. \textcolor{red}{É realmente necessário aplicar alguma mudança mais profunda, mas não sei ainda se ela sequer existe. Um caminho possível a partir daqui é analisar algo como a esperança sobre $V_0$ e $I_0$ a partir da distribuição e tal, e ai ter algo estatístico para falar: "olha, muito provavelmente não vai dar certo, escolha valores diferentes".}

De fato, a restrição do $\Delta_1$ é uma malha mais fina que a restrição de Sundman. Pela equação (\ref{eq:desigualdade_inercia_angular}) sabemos que $\tilde \sigma \geq \tilde c^2 I_0^{-1}$. Retomando a condição do $\Delta_1$ em (\ref{eq:restricao_delta1}), podemos aplicar esta desigualdade:
\begin{equation}\label{eq:desigualdade_generalizada}
    V_0^2 \geq \tilde \sigma (M^{-1} ||\tilde{\vet P}||^2 - 2 \tilde E)
    \geq \tilde c^2 I_0^{-1} M^{-1} ||\tilde{\vet P}||^2 - 2 \tilde E \tilde c^2 I_0^{-1}.
\end{equation}
Observe que o obtido aqui à direita é exatamente a condição de Sundman (\ref{eq:inequacao_sundman}) com um acréscimo do momento linear total desejado. A restrição $\Delta_1$, então é uma restrição mais forte que a de Sundman e deve ser levada em conta no condicionamento de valores iniciais.
 
\subsubsection{Momentos angular e linear totais nulos}\label{subsubsection:j_p_nulos}
Nesse caso, a coisa toda só depende do sinal de $\tilde E$, pois a expressão para $\beta$ fica:
\begin{equation}
    \beta^2 = \dfrac{\tilde E - \alpha V_0}{S_1}
    \Rightarrow
    \tilde E > \alpha V_0 \Rightarrow \alpha > \tilde E / V_0.
\end{equation}

Aqui, como em todo caso, existe um sem número de possibilidades de escolhas de $\alpha$ que dependem unicamente do propósito com a simulação. Podemos escolher, por exemplo,
\begin{equation}
    \alpha = 1 + \delta(\tilde E < 0) \tilde E / V_0.
\end{equation}
Uso isso no programa porque fica fácil de generalizar para o caso com momento linear. Uma outra escolha mais complicada seria por exemplo, tomando $s_{\tilde E} = sign(\tilde E)$, para $\tilde E \neq 0$ tomar
\begin{equation}
    \alpha = \dfrac{\tilde E}{V_0} (1 - k s_{\tilde E}),
    \quad
    \beta^2 = \dfrac{\tilde E - \tilde E (1 - k s_{\tilde E})}{S_1} = \dfrac{k s_{\tilde E} \tilde E}{S_1},
\end{equation}
e se $\tilde E = 0$ não mexer nas soluções:
\begin{equation}
    \alpha = 1, \quad \beta^2 = \dfrac{- V_0}{S_1}.
\end{equation}

Mais para frente vou fazer uma análise do impacto das escolhas nos resultados.


\subsubsection{Garantindo as outras integrais primeiras}
Para garantir as outras integrais primeiras, precisamos garantir que a restrição generalizada (\ref{eq:desigualdade_generalizada}) seja satisfeita.

Se o momento linear total desejado é nulo, é fácil ver que se $\tilde E \geq 0$ ou se $||\tilde{\vet J}|| = 0$, a condição é satisfeita de imediato, então não é necessário se preocupar. Em outros casos, basta verificar se a restrição generalizada está atendida.

Se o momento angular total desejado não for nulo e a restrição for atendida, podemos tomar como $\alpha$ o ponto que é mínimo para a parábola da inequação generalizada:
\begin{equation}
    \alpha^* = - \dfrac{V_0}{\tilde \sigma}.
\end{equation}
É verdade que qualquer $\alpha^*$ tomado entre as raízes da equação de restrição funcionará, então cabe estudar como cada uma impacta no problema. De repente no equilíbrio ou coisa assim, vamos ver.

Se por outro lado quisermos $||\tilde{\vet J}|| = 0$ mas $||\tilde{\vet P}|| \neq 0$, então podemos usar (\ref{eq:inequacao_delta1}) para obter valor de $\alpha^*$:
\begin{equation}
    \alpha > -\dfrac{||\tilde{\vet P}||^2}{2M V_0} - \tilde E / V_0.
\end{equation}
No caso, como o problema aqui pode aparecer devido ao segundo termo poder ser negativo de $\tilde E < 0$, então podemos novamente buscar $\alpha$ tal que $\tilde E - \alpha V_0 - S_2 > 0$, o que fornece:
\begin{equation}
    \alpha > -\dfrac{\tilde E}{V_0} - \dfrac{||\tilde{\vet P}||^2}{2 M V_0}.
\end{equation}
Seguindo a mesma ideia da primeira sugestão para quando os momentos totais são nulos, podemos tomar:
\begin{equation}
    \alpha^* = 1 + \delta(\tilde E < 0) \dfrac{\tilde E}{V_0} - \dfrac{||\tilde{\vet P}||^2}{2 M V_0}.
\end{equation}