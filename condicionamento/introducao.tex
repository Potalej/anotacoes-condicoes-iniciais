\section{Introdução}
A ideia inicial é a partir de um conjunto aleatório de posições e velocidades obter um novo conjunto de posições e velocidades que tenham determinadas integrais primeiras. Nesse caso aqui, não vou mexer nas massas.

Ah, de antemão vou considerar que no sorteio já consigo o centro de massas na origem. Isso é algo que preciso pensar depois, mas vou supor que já está feito.

Não é o objetivo aqui pensar em como obter valores iniciais antes do condicionamento. Talvez caiba escrever sobre no futuro, mas por agora vou supor que já tenho todos os valores iniciais em mãos e vou apenas manipulá-los.

\subsection{O que acontece se começamos com partículas sem velocidade?}
Se o sistema começa com energia cinética nula, as partículas vão para o centro e se intereferem mutuamente, ganhando momento angular em troca de perderem velocidade radial. Após o choque inicial, o sistema perde uma considerável quantidade de partículas que conseguem energia suficiente para escaparem, e o sistema segue para a relaxação violenta. Segundo testes de Standish em 1968, cerca de 15\% das partículas escapam. Experimentos de Lecar e Cohen (1972) em 1 dimensão também confirmam que a distribuição em equilíbrio só ocorre na parte principal (centro).

